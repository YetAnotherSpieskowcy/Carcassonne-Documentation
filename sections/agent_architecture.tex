\section{Agent Architecture}

The agent was implemented using Python due to the availability of useful libraries. 
To aid decision-making it is using a Monte Carlo Tree Search (MCTS) algorithm. 
It consists of the base module Tree used to manage the traversal process and 
the Node representing each state in the search tree. Together, these classes 
provide a basis for exploration and evaluation interfaces of possible game actions.

The Tree class initializes an agent with its initial game state and provides the 
communication with the environment using the GameDispatch object, which manages 
the underlying game engine. Tree is populated iteratively until one of the 
postconditions is reached (either time elapsed or maximum number of iterations). 
During the expansion phase in the MCTS algorithm, the Upper Confidence Bound (UCB) 
is used to select the most promising child nodes. This formula is used to obtain 
the balance between exploration of less-visited nodes and the ones that proved to yield highest rewards.

The core of the decision making is a combination of two evaluation metrics. 
One is calculated during selection (policy) and the other during expansion (value). 
As an input they take a serialized game state represented as a binary mask and both 
return respective evaluation scores reduced by discount factor, which is used to 
prioritize actions that provide the best short-term rewards during this process.

Finally, after exploration and expansion is completed the best action is determined 
by evaluation of the child nodes of the root. It is selected by calculating the 
average cumulative reward (Q value) of their corresponding subtrees. 
The result of the successful search is the most promising action chosen by 
the agent to be submitted as an actual move to the engine. 

This whole structure was inspired by the implementation of AlphaGo Zero, 
and it allows for an efficient and universal training process, since 
it is well suited for solving complex decision-making tasks in games like Carccassone. 
Based on this architecture, the team prepared agents using different 
tactics (e.g. greedy method and random selection) which later will be used to 
measure performance of a neural network agent and provide a baseline for comparison.


This article explores the use of artificial intelligence in board games, focusing
on the advancements of AlphaGo\cite{AlphaGoAlgorithm} and AlphaGo Zero\cite{AlphaGoZero}. 
AlphaGo marked a milestone in AI by defeating professional Go players using a 
combination of supervised learning and neural networks for policy and value estimation. 
AlphaGo Zero simplified the architecture by integrating policy and value evaluation 
into a single residual neural network and learning solely through self-play, 
eliminating reliance on human game data. Inspired by this approach, the development 
of a Carcassonne AI agent utilized a custom-built game engine designed for 
high computational efficiency and concurrency. The agent employs Monte Carlo Tree Search (MCTS) 
to navigate the game’s high branching factor and integrates policy and value-based 
evaluations for decision-making. Baseline strategies, such as greedy and random tactics, 
were implemented for performance comparison. The system design includes training 
environments and visualization tools, enabling detailed analysis of gameplay. 
This work lays a foundation for the development of an agent capable of professional play.

\small
\textbf{\textit{Keywords:}} machine learning, reinforcement learning, MCTS, board games
